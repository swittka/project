\section{Introduction}\label{introduction}
When designing an \emph{application-specific integrated circuit (ASIC)} it is
necessary to divide it into logical units called modules, so its complexity is
reduced to multiple manageable problems \cite{verilog}. Considering this, it is obvious to use the same
strategy when designing the verification environment of this chip. A commonly
used approach to get such a hierarchical testbench is to use the \emph{Univeral
Verification Methodology (UVM)} to divide the testbench into reusable
subcomponents. The resulting units are called \emph{Universal Verification
Components (UVCs)}. Such testbenches are written in special verification
languages like \emph{SystemVerilog}, \emph{e} or \emph{SystemC}.\\
Because the structure of an UVC supports reusability, it is typical that an UVC
is utilized again when its corresponding module is reused in another design. Now
it can occur that the test bench for the new design uses another verification
language than the reused UVC. In this case the previous created UVC cannot be
used and must be rewritten in the verification language of the new testbench.
This means, that additional time is needed to implement and test the UVC, which
actually already existed, just in another verification language. Therefore it
would be preferable to build a testbench consisting of UVCs regardless of the
verification language they are implemented in.\\
This is the idea behind  \emph{Universal Verification Language
Multi-Language (UVM-ML)}. It is an open source solution to combine
verification components written in different languages into a single
environment and its features are introduced in this report.



