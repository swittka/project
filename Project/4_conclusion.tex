\section{Conclusion}\label{conclusion}
So when facing the problem of having multiple verification components implemented in different frameworks, UVM-ML provides the ability to combine these components into a single verification hierarchy without rewritten some of the components from scratch. Many multi-language functionalities require minimal changes of the existing code. Especially building a unified hierarchy consisting of components implemented in different framworks can be done by adding a view lines of code. Similarly configuring the created hierachy through the configuration database is done without any major effort. Even establishing the communication between components using TLM ports, requires minimal changes of the existing code. Often components like a monitor already contain TLM ports, which makes it even simpler to set up the TLM communication by just connecting the existing interfaces.\\
Using a proxy sequencer to reuse foreign sequences is a convenient way of hiding the multi-language overhead, when implementing new sequences for the multi-language testbench. It allows the top level component to handle the foreign sequences like native ones. However enabling the reuse of foreign sequences takes a lot of effort caused by exporting each of the foreign sequences individually. But this is still negligible compared to the time needed to rewrite the whole component. The effort in reusing foreign sequences could be reduced by integrating a tool in the UVM-ML package, which generates the exported sequences based on the foreign sequences.\\
Furthermore the end of test synchronization should be done internally, just like the appropriate order of phase execution when building the unified hierarchy inside of the multi-language environment. Therefore the end of test would be reached, when all objections in each framework are dropped globally.\\
So concluding UVM-ML is a considerable way of reusing verification components of different frameworks to ensure a more efficient workflow, than rewriting parts of the components from scratch. UVM-ML version 1.5.1 enables the creation of a fully operative multi-language testbench, even if the features mentioned above could be integrated respectively improved.
Since the package is still further developed, it is likely, that additional features will be implemented respectively existing ones will be enhanced in the future.