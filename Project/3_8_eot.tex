\subsection{End of Test Synchronization}
At the moment UVM-ML does not provide any synchronization between frameworks to ensure the correct ending of a test. This means, if a framework reaches the end of the test, but another framework would need to execute additional code in its \lstinline$run$ phase, the first framework could preemtively terminate the whole test. This would prevent the other frameworks from finishing the test properly.\\
Until end of test synchronization is added to UVM-ML, it is recommended to exchange the status of objections manually. In doing so, the UVM-SV framework contains the master component ending the test, when all other slave components have dropped their objections. Example~\ref{lst:SV_sync_test_end} shows the implementation of the master component. The communication between master and slave components is established via a \lstinline$reg$ variable connected to a simple port in the UVM-\textit{e} component. Therefore a module is created containing this \lstinline$reg$ variable (line~\ref{line_sv_reg_sync}). In the \lstinline$run$ phase of the testbench the master raises its objection (line~\ref{line_sv_raise_obj}) and then waits for the slave component to drop its objections (line~\ref{line_sv_wait_e_obj}). Only after all objections at the slave component are dropped, the master drops its own objection (line~\ref{line_sv_drop_obj}) and the whole test can be stopped safely.
\lstset{language=SystemVerilog, numbers = left, escapechar=|, breaklines=true}
\begin{lstlisting}[frame=htrbl, caption={SystemVerilog: ending the test when UVM-\textit{e} components are ready},
label={lst:SV_sync_test_end}]
module topmodule;|\label{line_sv_reg_sync}|
  reg e_objections_dropped;
};

class tb extends uvm_test;
  task run_phase(uvm_phase phase);
    phase.raise_objection(this);|\label{line_sv_raise_obj}|
    wait_for_e_end_of_test();
  endtask : run_phase
  
  task wait_for_e_end_of_test();
    @topmodule.e_objections_dropped;|\label{line_sv_wait_e_obj}|
    uvm_test_done.drop_objection(this);|\label{line_sv_drop_obj}|
  endtask
endclass : tb
\end{lstlisting}
The end of test synchronization of a slave component is displayed in example~\ref{lst:e_slave_test_end}. It contains a simple port (line~\ref{line_e_simple_port}), which is connected to the \lstinline$reg$ variable of the SystemVerilog module in the previous example. When all objections are dropped the method \lstinline$all_objections_dropped()$ is automatically called. It is overwritten in line~\ref{line_e_objections_dropped} to assert the prior defined simple port. Thus the master component gets indicated, that it can end the test. If the end of test synchronization is implemented in a subcomponent, it is mandatory to disable the method \lstinline$all_objections_dropped()$ of \lstinline$sys$. It calls \lstinline$stop_run()$, which ends the test without any synchronizations between the frameworks.
\lstset{language=e, numbers = left, escapechar=|, breaklines=true}
\begin{lstlisting}[frame=htrbl, caption={\textit{e}: signaling master component that all objections are dropped},
label={lst:e_slave_test_end}]
extend sys {
  all_objections_dropped_op : out simple_port of bit is instance;|\label{line_e_simple_port}|
    keep bind(all_objections_dropped_op, external);
    keep all_objections_dropped_op.hdl_path() == "~/topmodule.e_objections_dropped";
    
  all_objections_dropped(kind: objection_kind) is only {|\label{line_e_objections_dropped}|
    if kind == TEST_DONE {
      all_objections_dropped_op$ = 1;
    };
  };
};
\end{lstlisting}