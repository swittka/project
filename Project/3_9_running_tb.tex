\subsection{Running a Test in a Multi-Language Environment with Incisive Enterprise Simulator}

After completing the testbench the final step is to run the multi-language environment. Here is shown how to achieve
this with Incisive Enterprise Simulator provided by Cadence Design Systems Inc. Also a task of the
UVM-SV adapter is presented for this purpose, which provides an alternative option for running a test.

\subsubsection{Setup IES for UVM-ML Mode}
Independent of how the test is started, IES has to be set up to contain the functionalities necessary to support multi-language testbenches. By the use of at least one of the irun invocation options \lstinline$-ml_uvm$, \lstinline$-uvmtop$ or \lstinline$-uvmtest$ the UVM-ML mode of Incisive is turned on. Thereby \lstinline$-ml_uvm$ has only the purpose to enable the multi-language mode and the other two options additionally have the in section~\ref{uvm_top} explained functionalities.\\
With the installation of the UVM-ML package multiple argument files are created, which can be used to configure IES. They are located at the directory:
\lstset{language={}, numbers=none, escapechar=|}
\begin{lstlisting}
$UVM_ML_HOME/ml/run_utils/
\end{lstlisting}
Each argument file is available in a 32 bit and a 64 bit version, so the appropriate one can be used for the UVM-ML installation (just insert the suitable version number for \emph{XX} in the following file names). A file with general UVM-ML options is available, which already contains \lstinline$-ml_uvm$ to ensure the start of IES in multi-language mode called \lstinline$ml_options.XX.f$. Additional argument files for each framework were created, which can be invoked in \lstinline$irun$:
\begin{itemize}
  \item UVM-\textit{e}: \lstinline$e_options.XX.f$
  \item UVM-SV 1.1: \lstinline$sv_1.1_options.XX.f$
  \item UVM-SV 1.2: \lstinline$sv_1.2_options.XX.f$
  \item UVM-SystemC: \lstinline$sc_options.XX.f$
\end{itemize}
\subsubsection{Starting a test with the irun utility} \label{uvm_top}
IES provides command line switches for its \lstinline$irun$ utility to declare the top component(s) of the testbench as
well as the to be started test as shown below.
\medskip
\lstset{language={}, numbers=none, escapechar=|}
\begin{lstlisting}
-uvmtest frmw_indicator:test_name
-uvmtop frmw_indicator:top_name
\end{lstlisting} 
\medskip
Both \lstinline$uvmtest$ as well as \lstinline$uvmtop$ firstly need to identify the framework adapter of the test
respectively top component. This is done via \lstinline$frmw_indicator$ which can be \lstinline$SV$, \lstinline$e$ or
\lstinline$SC$ as described in section~\ref{sv_inside_e}. Seperated via a colon comes the actual test/top component. For
UVM-\textit{e} this is the full path of the file containing it. In contrast UVM-SV along with UVM-SystemC uses
its component type name. So contrary to UVM-\textit{e} the full path to the file containing it has to be forwarded to
\lstinline$irun$ prior to that.

\subsubsection{Starting a test via the UVM-SV adapter}

Alternativly to these command line switches the UVM-SV adapter supports a procedure call to set the top
component(s) respectively the test of the multi-language testbench. The other adapters for UVM-\textit{e} and
UVM-SystemC do not support such a call. So there is the only possibility to use the above mentioned command line
switches. The task provided by the UVM-SV adapter is called \lstinline$uvm_ml_run_test$ and has the following
syntax:
\medskip
\lstset{language={}, numbers=none, escapechar=|}
\begin{lstlisting}
task uvm_ml_run_test(
  string tops[],
  [string test = ""])
\end{lstlisting} 
\medskip
Where \lstinline$tops[]$ is a dynamic array of top component identifiers. Optionally \lstinline$test$ represents a
UVM-SV or UVM-SystemC test component or the full path of a file containing an UVM-\textit{e} test unit. Both arguments
support the same syntax as the command line switches to specify the components (see section~\ref{uvm_top}). The argument
indicating the test component is optional for the purpose of just generating the multi-language environment.