\subsection{Integrate Multi-Language Functionality into Incisive Enterprise
Simulator}
At the current state of work multi-language support can be integrated into Cadence Incisive Enterprise Simulator via an open source package provided by Accellera Systems Initiative. It is developed jointly by AMD and Cadence Design
Systems. This package is called \emph{UVM-ML Open Architecture} and is provided as download on the Accellera website \cite{uvm_ml}.
For this report version 1.5.1 of the package was used. After downloading and extracting, it can be
installed via the provided install script called \lstinline$install_ies.csh$ and the following steps using \emph{Bourne shell}:

\medskip
\lstset{language={}, numbers=none, escapechar=|}
\begin{lstlisting}
% export UVM_ML_HOME=<install_dir>
% source /opt0/Administration/eda/cadence_path.INCISIV142.linux
% csh -c "source $UVM_ML_HOME/ml/install_ies.csh [--64bit] [--no-osci]"
\end{lstlisting} 
\medskip 

Due to some problems with the installation on \emph{Ubuntu} it is recommended to use an alternative distribution like
\emph{CentOS}. The environment variable \lstinline$UVM_ML_HOME$ has to point to the root directory of the UVM-ML
package, which is \lstinline$UVM_ML-1.5.1$ for version 1.5.1. Within the \emph{Computer Architecture Group} the standard setup script can be used
to set the path environment variable to the \emph{irun} installation. While sourcing
\lstinline$install_ies.csh$ additional options can be used to control it. To install the environment on a 64 bit machine
instead of a 32 bit one use \lstinline$--64bit$. If only the adapters for UVM-SV, UVM-SystemC and
UVM-\textit{e} are required, UVM-ML can be build without the ASI-SystemC adapter via \lstinline$--no_osci$.\\
After this installation UVM-ML enables the creation of multi-language environments as well as running tests, which is
discussed in the following sections.