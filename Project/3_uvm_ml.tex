\section{Universal Verification Methodology Multi-Language}\label{uvm_ml}
- why use UVM-ML?\\
- benefits of UVM-ML\\
\\
- structure for each subsection:\\
- how does it work\\
- what do i need to change in the code\\
- code example\\
\\
\subsection{Integrate Multi-Language Functionality into Incisive Enterprise
Simulator}
- where to get UVM-ML\\
- how to install\\
- problems with installation\\
\\
\subsection{Creating a Multi-Language Environment}
When planing to create a multi-language environment, there are two possible
approaches supported by UVM-ML to achive this goal. Firstly an \emph{unified
hierachy} can be created or alternatively an \emph{side-by-side} environment. In an unified
hierachy a child component implemented in one verification language is
instantiated from a parent component in another verification language. In
contrast, when creating a side-by-side architecture, the environment contains
multiple tops.

\subsubsection{Instantiating a SystemVerilog Component within an e Unit}

\subsubsection{Instantiating an e Unit Within a SustemVerilog Component}

\subsubsection{Creating a Side-by-Side Environment}
- instantiate e UVC in sv environment\\
- instantiate sv UVC in e environment\\
- side-by-side \\
- ??? already rate the different types (pros/cons) or in conclusion?\\
\\
\subsection{Configuring a Multi-Language Environment}
- distribute config across the subcomponents\\
\\
\subsection{Data Communication in a Multi-Language Environment} \label{ml_tlm}
- tlm-port e -> sv\\
- tlm-port sv -> e\\
- type mapping\\
\\
\subsection{Sequence Layering in a Multi-Language Environment}
- adding tlm interface\\
- create proxy sequencer\\
- doing sequence items\\
- doing sequences\\