\subsection{Data Communication in a Multi-Language Environment} \label{ml_tlm}
Additionally to the configuration database, UVM-ML uses TLM communication to manage the communication between components
of different frameworks. While communication via the configuration database only occurs once per item to build the
structure of the testbench, TLM communication is used to establish ongoing communication between components during the
test. Data transfer by use of UVM-ML TLM communication is achieved by serializing transactions. After that, the resulting
bitstream is sent to the destination component via the connected port. Eventually it is de-serialized there to regain the transaction.\\
For this it is required, that the backplane knows, which types are mapped onto each other. Also the serialization and
de-serialization need to match, so the same transaction is available at both components. This implies, that the number
and order of fields in the user defined types have to be consistent.
Here it is discussed how to establish TLM communication on the example of TLM analysis ports. For more information on
type mapping see section~\ref{type_mapping}.\\
When it is ensured, that both frameworks handle compatible data items, TLM analysis ports have to be implemented in each
component including the \lstinline$write()$ method of the receiving TLM port. After the ports are instantiated they
are registered at their framework adapter. For UVM-SV this is done by calling
\lstinline$uvm_ml::tlm1::register()$ and for UVM-\textit{e} the port is bind to external by constraining \lstinline$bind()$.
Finally the ports are connected via \lstinline$uvm_ml::connect()$ for UVM-SV or
\lstinline$uvm_ml.connect_names()$ for UVM-\textit{e}. This should only be done once, either at the
SystemVerilog component or the UVM-\textit{e} unit.
\begin{itemize}
\item{UVM-SV:}
%\medskip
\lstset{language={}, numbers=none, escapechar=|}
\begin{lstlisting}
static function void uvm_ml::ml_tlm1#(T)::register(port);

function bit uvm_ml::connect(
  string producer_path,
  string consumer_path);
\end{lstlisting} 
%\medskip

\item{UVM-\textit{e}}
%\medskip
\lstset{language={}, numbers=none, escapechar=|}
\begin{lstlisting}
keep bind(port, external);

uvm_ml.connect_names(
  producer_path : string,
  consumer_path : string) : bool;
\end{lstlisting} 
%\medskip
\end{itemize}

Following, examples for sending transactions from UVM-\textit{e} units to UVM-SV components and
vice versa are provided. They show, how to utilize the above introduced TLM functionalities for multi-language
environments.
\subsubsection{\textit{e} Units Sending Transactions to UVM-SV Components}
In example~\ref{lst:SV_TLM_consumer} is the implementation of a TLM implementation port in UVM-SV
displayed. After declaring (line~\ref{line_sv_imp_declare}) and instantiating (line~\ref{line_sv_imp_inst}) the port, it
is registered at the UVM-SV multi-language adapter (line~\ref{line_sv_imp_reg}).
\lstset{language=SystemVerilog, numbers = left, escapechar=|, breaklines=true}
\begin{lstlisting}[frame=htrbl, caption={SystemVerilog: consumer side of a TLM connection},
label={lst:SV_TLM_consumer}]
class mod_monitor extends uvm_monitor;
  uvm_analysis_imp#(packet, mod_monitor) in_port;|\label{line_sv_imp_declare}|
  
  `uvm_component_utils(mod_monitor)
  
  function new(string name="mod_monitor", uvm_component parent);
    super.new(name, parent);
    in_port = new("in_port", this);|\label{line_sv_imp_inst}|
  endfunction 
  
  function void phase_ended(uvm_phase phase);
    if(phase.get_name() == "build") begin
      uvm_ml::ml_tlm1#(packet)::register(in_port);|\label{line_sv_imp_reg}|
    end
  endfunction : phase_ended
  
  virtual function void write(packet pkt);|\label{line_sv_imp_write}|
  //...
  endfunction : write
endclass
\end{lstlisting}
The corresponding UVM-\textit{e} producer is shown in example~\ref{lst:e_TLM_producer}. Its TLM port is implemented in
line~\ref{line_e_port_inst} and following registered by constraining \lstinline$bind()$ in line~\ref{line_e_port_reg}.
After both components are instantiated under \lstinline$sys$, their TLM ports are connected in line~\ref{line_e_connect}.
\lstset{language=e, numbers = left, escapechar=|, breaklines=true}
\begin{lstlisting}[frame=htrbl, caption={\textit{e}: producer side of a TLM connection},
label={lst:e_TLM_producer}]
unit inf_monitor like uvm_monitor {
  out_port : out interface_port of tlm_analysis of packet is instance;|\label{line_e_port_inst}|
    keep soft bind(out_port, external);|\label{line_e_port_reg}|
};

extend sys {
  e_inf_monitor  : inf_monitor is instance;
  sv_mod_monitor : child_component_proxy is instance;
    keep sv_mod_monitor.type_name == "sv:mod_monitor";
  
    
  connect_ports() is also {
    uvm_connect_names("sys.e_inf_monitor.outport", append(sv_mod_monitor.e_path(), ".in_port"));|\label{line_e_connect}|
  };  
};
\end{lstlisting}

\subsubsection{SystemVerilog Components Sending Transactions to UVM-\textit{e} Units}
Now the functionalities of consumer and producer are switched between both frameworks, so the TLM implementation port is
realized in UVM-\textit{e} (example~\ref{lst:e_TLM_consumer}). It is instantiated in line~\ref{line_e_imp_inst} followed
by its registration in line~\ref{line_e_imp_reg}.
\lstset{language=e, numbers = left, escapechar=|, breaklines=true}
\begin{lstlisting}[frame=htrbl, caption={\textit{e}: consumer side of a TLM connection},
label={lst:e_TLM_consumer}]
unit mod_monitor like uvm_monitor {
  in_port : in interface_port of tlm_analysis of packet is instance;|\label{line_e_imp_inst}|
    keep soft bind(out_port, external);|\label{line_e_imp_reg}|
    
  write(pkt : packet) is {|\label{line_e_imp_write}|
  //...
  };
};
\end{lstlisting}
The producer is shown in example~\ref{lst:SV_TLM_producer}. Its TLM port is declared in line~\ref{line_sv_port_declare}
and instantiated in line~\ref{line_sv_port_inst}. After that it is registered at the framework adapter in
line~\ref{line_sv_port_reg}. Finally both monitors are instantiated (line~\ref{line_sv_inst_monitors}) and their ports are connected (line~\ref{line_sv_con}).
\lstset{language=SystemVerilog, numbers = left, escapechar=|, breaklines=true}
\begin{lstlisting}[frame=htrbl, caption={SystemVerilog: producer side of a TLM connection},
label={lst:SV_TLM_producer}]
class inf_monitor extends uvm_monitor;
  uvm_analysis_port#(packet) out_port;|\label{line_sv_port_declare}|
  
  `uvm_component_utils(inf_monitor)
  
  function new(string name="inf_monitor", uvm_component parent);
    super.new(name, parent);
    out_port = new("out_port", this);|\label{line_sv_port_inst}|
  endfunction 
  
  function void phase_ended(uvm_phase phase);
    if(phase.get_name() == "build") begin
      uvm_ml::ml_tlm1#(packet)::register(out_port);|\label{line_sv_port_reg}|
    end
  endfunction : phase_ended
endclass

class env extends uvm_env;
  inf_monitor   sv_inf_monitor;
  uvm_component e_mod_monitor;
  
  `uvm_component_utils(env)
  
  function void build_phase(uvm_phase phase);|\label{line_sv_inst_monitors}|
    super.build_phase(phase);
    sv_inf_monitor = inf_monitor::type_id::create("sv_inf_monitor", this);
    e_mod_monitor  = uvm_ml_create_component("e", "mod_monitor", "e_mod_monitor", this); 
  endfunction : build_phase
  
  function void connect_phase(uvm_phase phase);
    uvm_ml::connect(sv_inf_monitor.get_full_name(), {get_full_name(),".e_mod_monitor.in_port"});|\label{line_sv_con}|
    endfunction : connect_phase
\end{lstlisting}